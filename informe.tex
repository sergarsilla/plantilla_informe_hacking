\documentclass[12pt, a4paper]{article}

% --- PAQUETES ---
\usepackage[T1]{fontenc}      % Para que el PDF use fuentes con acentos.
\usepackage{lmodern}          % Carga una fuente moderna compatible con T1.
\usepackage[utf8]{inputenc}   % Para que LaTeX lea los acentos de tu .tex.
\usepackage[spanish]{babel}   % Activa las reglas del idioma español.
\usepackage{geometry}
\usepackage{graphicx}
\usepackage{hyperref}
\usepackage{xcolor}
\usepackage{listings}
\usepackage{float}
\usepackage{titlesec}

% --- CONFIGURACIÓN ---
\geometry{a4paper, margin=2.5cm}
\hypersetup{
    colorlinks=true,
    linkcolor=black,
    urlcolor=black,
    citecolor=black
}
\graphicspath{{images/}} % Ruta a la carpeta de imágenes

% Espaciado entre párrafos y sin sangría
\setlength{\parskip}{1em}
\setlength{\parindent}{0pt}

% Más espacio antes/después de secciones y subsecciones
\titlespacing*{\section}{0pt}{2em}{1em}
\titlespacing*{\subsection}{0pt}{1.5em}{0.8em}

% Limita la profundidad del índice a secciones y subsecciones (Nivel 2)
\setcounter{tocdepth}{2}

% --- ESTILO PARA BLOQUES DE CÓDIGO ---
\definecolor{backcolour}{rgb}{0.95,0.95,0.95}
\lstdefinestyle{mystyle}{
    backgroundcolor=\color{backcolour},
    basicstyle=\footnotesize\ttfamily,
    breaklines=true,
    numbers=left,
    numbersep=5pt,
    numberstyle=\tiny\color{gray},
    showstringspaces=false,
}
\lstset{style=mystyle}

% --- DATOS DE PORTADA (PLANTILLA) ---
\title{
    \vspace{2cm}
    \textbf{Informe de Hacking Ético} \\
    \large \textit{Subtítulo1 --- Subtítulo2}
    \vspace{3cm}
}
\author{Autor: [TU NOMBRE COMPLETO]}
\date{\today}


% --- INICIO DEL DOCUMENTO ---
\begin{document}

\maketitle
\thispagestyle{empty}
\newpage

\tableofcontents
\newpage

% --- INCLUSIÓN DE LAS SALAS RESUELTAS ---
% Para añadir una room, copia 'plantilla_room.tex', renómbrala
% y añade un \input aquí con el nuevo nombre.

% =============================================================
% ===      PLANTILLA PURA PARA RESOLUCIÓN DE ROOM
% =============================================================

\section{[NOMBRE DE LA SALA]}

% -------------------------------------------------------------------
\subsection{Fase 1: Reconocimiento y Enumeración}
% -------------------------------------------------------------------
[Breve descripción del proceso de escaneo y los hallazgos principales.]

\textbf{Comando(s) ejecutado(s):}
\begin{lstlisting}[language=bash]
[Pega aqui tu(s) comando(s) de Nmap, GoBuster, etc.]
\end{lstlisting}

\textbf{Resultados / Evidencia:}
%
% [AQUÍ VA LA CAPTURA DE PANTALLA DEL ESCANEO]
% Para añadirla, usa: \includegraphics[width=0.9\textwidth]{nombre_de_tu_captura.png}
%

% -------------------------------------------------------------------
\subsection{Fase 2: Explotación y Acceso Inicial}
% -------------------------------------------------------------------
[Descripción de la vulnerabilidad encontrada y cómo se va a explotar.]

\begin{enumerate}
    \item Primer paso del proceso de explotación...
    \item Segundo paso...
\end{enumerate}

\textbf{Comando(s) de explotación:}
\begin{lstlisting}[language=bash]
[Pega aqui el comando del exploit, la reverse shell, etc.]
\end{lstlisting}

\textbf{Evidencia de acceso (Shell):}
%
% [AQUÍ VA LA CAPTURA DE PANTALLA DE LA SHELL]
%

% -------------------------------------------------------------------
\subsection{Fase 3: Post-Explotación y Escalada de Privilegios}
% -------------------------------------------------------------------
[Explicación del vector de escalada de privilegios que has descubierto.]

\textbf{Comando(s) de escalada:}
\begin{lstlisting}[language=bash]
[Pega aqui el comando que te dio acceso root.]
\end{lstlisting}

\textbf{Evidencia de escalada (whoami/id):}
%
% [AQUÍ VA LA CAPTURA DE PANTALLA DEMOSTRANDO ACCESO ROOT]
%

% -------------------------------------------------------------------
\subsection{Fase 4: Captura de Evidencias (Flags)}
% -------------------------------------------------------------------
\textbf{Bandera de Usuario (user.txt):}
%
% [AQUÍ VA LA CAPTURA DE PANTALLA DE LA BANDERA DE USUARIO]
%

\textbf{Bandera de Root (root.txt):}
%
% [AQUÍ VA LA CAPTURA DE PANTALLA DE LA BANDERA DE ROOT]
%

% -------------------------------------------------------------------
\subsection{Fase 5: Resumen y Recomendaciones}
% -------------------------------------------------------------------
[Breve resumen del vector de ataque completo y las recomendaciones de seguridad para mitigar las vulnerabilidades encontradas.]

\newpage % Cambia esto por el nombre de tu primera sala
% \input{rooms/nombre_de_tu_segunda_sala.tex}


\end{document}